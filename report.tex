\documentclass[a4paper, 12pt]{report}
\usepackage[utf8]{inputenc}
\usepackage[swedish,british]{babel}
\usepackage{fancyhdr}
\usepackage{listings} 

\pagestyle{fancy}
\title{Integer Factorization}
\author{Diana Gren \\ Jonas Carlsson \\\\ DD2440 Advanced Algorithms \\ Stefan Nilsson}
\date{}

\begin{document}
\maketitle
\tableofcontents
\newpage
\chapter{Background}
Factorize integers into prime numbers is considered a difficult problem. There is algorithms that solve the problem, but they can in some cases be very time consuming. An example is a product of two large prime numbers, which is very difficult to factorize. This is the key in RSA cryptography, which completely relies on the fact that factoring products of large prime numbers is incredibly difficult.

\section{Algorithms}
As mentioned above, there are some algorithms that solve the problem, of which two of them will be discussed in this section.

\subsection{Trial Division}
\subsubsection{Method}
The most straight forward way of solving this problem is of course to divide a given number \emph{N} with each prime number up to $ \sqrt{N} $. If we find a prime $x$ such that
\begin{equation}
x \equiv 0 \pmod b
\end{equation}
we have found two factors. The first factor being $x$ and the second being $ \frac{N}{x} $.

\subsubsection{Performance}
Trial division is not an ideal approach as it is a heavy algortihm \cite{trial}. It can, however, find a small factor fast if such a factor exists. Since there is a 50\% chance that a large $N$ is even, and 33\% that it has a factor 3 and so on, it can be a good idea to check for small prime numbers as factors.

\subsection{Fermat's Factorization Method}
\subsubsection{Method}
A simple way of improving the naive method is to use the fact that every odd number number $N$ can be represented as a difference of squares (proof in section \ref{sec:fermatproof}.
\begin{equation}
N = a^2 - b^2
\end{equation}
If we know $a$ and $b$, $N$ can easily be factored into two factors according to mathematical laws
\begin{equation}
N = a^2 - b^2 = (a + b)(a - b)
\end{equation}

To find such integers, one starts with an $x_0 = \lceil \sqrt{N} \rceil $. Check if 
\begin{equation}
\Delta x_0 = x_0^2 - N
\end{equation}
is a square number. If it is not, try next; $x_1 = x_0 + 1$
\begin{equation}
\Delta x_1 = x_1^2 - N
\end{equation}
etc. Do the same step, increasing $x$ by one, until a square number $ \Delta x $ is found.

\subsubsection{Performance}
Fermat's factorization method works well when the resulting factors are of similar size \cite{mathdep}. If that is not the case, this method is not optimal. However, the method sets base for more advance and better performing algorithms such as the \emph{quadratic sieve} and \emph{general number field sieve}.

\subsection{Pollard Rho}
The Pollard Rho algorithm was invented by John Pollard in 1975, and its speciality is to factor composite numbers with small factors.

\subsubsection{Method}
The idea with the Pollard Rho method is to iterate over the ring $mod(N)$ until falling into a cycle. If $N = pq$, where $q$ and $q$ are two prime factors, we can produce a sequence if numbers that eventually fall into a cycle by iterating a polynomial formula \cite{pollard}. \\
Ex 
\begin{equation}
x_{n+1} = (x_n^2 + a) \bmod n
\end{equation} \\
According to the \emph{Chinese remainder theorem} (section \ref{sec:chinese}), we know that each value of $x \bmod n$ corresponds to a value in the pair $x \bmod p$ and $x \bmod q$, since $p$ and $q$ are primes, hence also relatively prime. 
\\
This leads to the sequence of numbers $x_n$:s follows the same formula modulo $p$ and $q$:
\begin{equation}
	x_{n+1} = ((x_n \bmod p)^2 + a) \bmod p
	x_{n+1} = ((x_n \bmod q)^2 + a) \bmod q
\end{equation}

The expected time until the $x_n$:s become cyclic and the expected cycle length are proportinal to $\sqrt{N}$, hence the sequence $(\bmod\,p)$ will fall into a much shorter cycle with an expected length of $\sqrt{p}$
\\
To find the prime factor, one needs to find the greatest common divisor which is equal to $p$.
\begin{equation}
gcd(|x_{n+1} - x_n|, N) = p
\end{equation}

\subsubsection{Performance}
The algorithm is based on the Birthday Problem (section \ref{sec:birthday}) and as was the case in Trial Division, Pollard rho is also fast in finding small factors \cite{pollard}. A critical part of the Pollard rho method is the detection of a cycle, i.e. if the sequence has become periodic, which is the time to stop the algorithm. In basic Pollard $x_i$ is compared to $x_{2i}$ for all $i$. An alternative method is Brent's factorization method which uses another kind of different cycle detection. The Pollard rho method can be significantly improved by implementing Brent's variant.

\subsection{Pollard Rho with Brent}
Brent published this improvement in 1980\cite{brentpdf} and it bases itself on improving cycle finding.
\subsubsection{Method}
Pollard-Rho is a great algorithm but it suffers from one flaw. It can't handle a cyclic behaviour. With a cyclic behaviour i mean the function f produces repeated results. We can take an example from our own pollard-rho function. When we tried to factor 25 we got this sequence:
\begin{center}
$a_1=1, a_2=22, a_3=16, a_4=2, a_5=19, a_6=7, a_7=2, a_8=19...$
\end{center}
Where a is the value to be investigated with GCD. This sequence will go nowhere and we will have to abort this run.

Cycle finding can be implemented with either Floyds "tortoise and hare algorithm" or with Brents version of Floyd. We start by discussing Floyds algorithm. This algorithm builds upon the idea that for integers $i >= \mu$ and $k >= 0, x_i = x_{i + k\lambda}$, where $\lambda$ is the length of the loop. Especially when $i = k\lambda >= \mu$, it follows that $x_i = x_{2i}$. Thanks to this we can find a length $v$ by letting one value take one step and the other take two steps. This value $v$ is divisable with $\lambda$. Which is what we are searching for. Brent realised that if we looked at the comming $2^i$ following numbers after $x_{2^i-1}$ would find the correct cycle length faster. Although it does not differ much for finding factors Brents algorithm finds the correct cycle length directly instead of having to search again like in Floyds. It also has to evaluate the function f one less time.

\subsubsection{Performance}
This algorithm as said before builds upon floyds. Floyds have a worst case of $O(\lambda + \mu)$. According to Brent his code would work 36\% percent faster for general cycle finding and 24\% for factorisation problems.

\chapter{Approach}
We decided to do the programming in C++. We both had limited knowledge in the language and wanted to learn more, as well as handling I/O for Kattis felt like a much easier task in C++ than in Java. 

We chose to implement the Pollard Rho algorithm, since it is not too difficult to understand nor to implement. The objective was to start with the very basic Pollard rho and optimize it with different small tweaks, and in the end also implement Brent's method.

Since Fermat's method seemed to not perform as well, we chose to discard both that and trial division. Some reading was made on the quadratic sieve, but it seemed well too complicated to understand and implement within the time restrictions of this project.

\section{Pollard Rho}
\subsection{Implementation}

We quickly realized that we needed the program to stop looking for factors if it was taking too long, so we introduced a cut off limit. The cut off limit is a limit for the maximum number of iterations the algorithm is allowed to do, and gives the answer "fail" if it did not succeed in fatorizing the given number. 

The scores could be improved just by increasing the cut off limit so that the program would run for as close to 15 seconds as possible, since it obviously had time to evaluate more numbers.

\subsection{Optimizations}
We recognized that since there is a 50/50 chance that the given number is even, it could be a good idea to check that before running the algorithm, and thereby eliminate calculation time. So we started off by introducing a check for even numbers, and just return true if the condition was satisfied.

After reading more on Pollard Rho, we found that finding the greatest common divisor is quite expensive to do, hence we do not want to do that so often. This made us multiply the value a certain number of times before entering the gcd, hoping to improve the algorithm.

\section{Pollard Rho with Brent}
\subsection{Implementation}
When working on this we realised we could not use the old way of cutting the program as it hade more parts to it. So we had to rewrite it to give consistent breaks. We wrote a method to check how many milliseconds had passed since the program started. Further optimizing this it was found that the random function we used was a huge time sink and choosing the values in a predetermined way was more effective in general, it was also noticed that in choosing this numbers we could greatly improve or decrease performance indicating that in choosing starting values can improve results even further.

\subsection{Problems}
Finding information on how to implement pollard-rho with brent proved hard as the general information we found was for the cycle finding algorithm. We did not find any information to the combined version of pollard-brent at first. But later we found the the paper that brent himself published. Which resulted in the current iteration of our code\cite{brentpdf}.
\chapter{Results}

\section{Pollard $\rho$}
The basic Pollard $\rho$ algorithm performed acceptably, and received a score of 39 on Kattis Judge. The best performance was achieved by tweaking the value of the cut off limit.
\begin{table}[ht]
\caption{Performance of Pollard $\rho$ depending with different cut off limits.}
\begin{tabular} {c c c}
Cut off limit 	&	 Score 	& Time [s] \\ \hline
130		&	12		& 1.2 \\
1000	&	15		& 2.4 \\
10000 	& 	32		& 12.1 \\
12500	&	39		&14.20 \\

\end{tabular}
\end{table}

In addition to the cut off limit, a table with the first 20 prime numbers was implemented. With this used, the score on Kattis increased to 41. The table was used for a trial division of a given number, before entering the Pollard algorithm.

\section{Brent}
Successfullt implementing Brent's algorithm showed to be much more difficult than expected. However, after a semi successful implementation was made, a score of 51 was received from Kattis.

\chapter{Conclusion}
Since we had so much trouble implementing a successful Brent algorithm, we did not achieve a high score. But when that implementation finally worked, it gave us a score much higher than the basic pollard without any further optimizations.

By tweaking the pollard algorithm, we could increase the score significantly, hence a legitimate assumption would be that we probably would have been able to optimize the Brent algorithm as well. 

Conclusion; Brent seems to be a very good approach for this project, as it is not too complicated for a masters student, such as yours truly, to understand and fairly easy to implement even though we had some trouble.
\appendix
\chapter{Theorems}
\section{Fermat Difference of Squares}
\label{sec:fermatproof}
\begin{description}
\item{\bf Theorem:} An odd integer $N$ can be represented as a difference of squares $N = a^2 - b^2$.


\item{\bf Proof:} \\
Let $ N = m_1m_2 $, with $ m_1 \le m_2 $. Since we know that $N$ is odd, $m_1$ and $m_2$ must both be odd as well.

Let $ a = \frac{1}{2} (m_2 + m_1) $ and $ b = \frac{1}{2} (m_2 - m_1) $. Since both $m_1$ and $m_2$ are odd, both $a$ and $b$ will be integers. 

This gives us $m_1 = a - b$ and $m_2 = a + b $, hence $N = m_1m_2 = (a - b)(a + b) = a^2 - b^2$.

\end{description}

\section{Chinese Remainder Theorem}
\label{sec:chinese}

\begin{description}
\item{\bf Theorem:} Let $x$ and $y$ be two relatively prime positive integers, and $a$ and $b$ be any two integers. Then there is an integer $N$ such that 
\begin{equation}
N \equiv a \pmod x 
\end{equation}
and
\begin{equation}
N \equiv b \pmod y
\end{equation}
I.e. $N$ is uniqely determined modulo $r,s$.

\end{description}

\section{Birthday Problem}
\label{sec:birthday}
The Birthday Problem concerns the probability that some pair in a set of $n$ randomly chosen people will have the same birthday.

\chapter{Code}
\section{Pollard $\rho$}
\begin{lstlisting}[frame=single] 
f(z, N)
	return z*z % N
function pollard(N) 
	x := random(N) 
	y := x 	
	prod := 1

	for i := 1 to infinity do
		x = f(x, N)
		y = f(f(y, N), N)

		if (x - y = 0) then
			prod = prod *(x-y) % N
			d := gcd(N, prod)

		if d > 1 and d < N then
			return d

		if i >= CUT_OFF_LIMIT then
			return 0
	end for
end function
\end{lstlisting}

\section{Pollard $\rho$ with Brent}
% This looks horrible... must fix in some way
\begin{lstlisting}[frame=single]
f(z, N)
   return z*z % N
function pollard_breant(N)
   y:=x0, r:=1, q:=1;
   while (G = 1)
      x=y;
      for i:=1 to r do y:=f(y,N), k:= 0
         while (k>=r) or (G>1)
            ys := y
            for i:=0 to min(m, r-k) do
               y:= f(y)
               q = (q * |x-y|) mod N
            end
         end
         G := GCD(q,N); k:=k+m
      end
      r=r*2
   end
   if G = N then
      while g = 1 do
         ys = f(ys,N); G= GCD(|x-y|, N)
      end
   end
   if G = n then success else fail
end
\end{lstlisting}
\begin{thebibliography} {10}
\bibitem{wolfram}
	WolframMathWorld.
	\emph{Fermat's Factorization Method} \\
	http://mathworld.wolfram.com/FermatsFactorizationMethod.html \\
	Retreived: Oct 2012
	
\bibitem{mathdep}
	Department of Mathematics, Kansas State University.
	\emph{Factoring Methods} \\
	http://www.math.ksu.edu/math511/notes/925.html \\
	Retreived: Oct 2012
	
\bibitem{mathdep2}
	Math \& Statistics Solutions.
	\emph{Fermat's Factoring Technique} \\
	http://mathsolutions.50webs.com/fermat.html \\
	Retreived: Oct 2012

\bibitem{trial}
	Wikipedia.
	\emph{Trial Division} \\
	http://en.wikipedia.org/wiki/Trial\_division \\
	Retreived: Oct 2012

\bibitem{chinese}
	WolframMathWorld.
	\emph{Chinese Remainder Theorem} \\
	http://mathworld.wolfram.com/ChineseRemainderTheorem.html \\
	Retreived: Oct 2012

\bibitem{pollard}
	WolframMathWorld.
	\emph{Pollard rho Factorization Method} \\
	http:http://mathworld.wolfram.com/PollardRhoFactorizationMethod.html \\
	Retreived: Oct 2012

\bibitem{brentpdf}
	Brent, Richard P. An Improved Monte Carlo Factorization Algorithm \\
	http://wwwmaths.anu.edu.au/~brent/pd/rpb051i.pdf \\
	Published 1980, Bit Numerical Mathematics Volume 20 Number 2

\end{thebibliography}
\end{document}